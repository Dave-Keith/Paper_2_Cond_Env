% Options for packages loaded elsewhere
\PassOptionsToPackage{unicode}{hyperref}
\PassOptionsToPackage{hyphens}{url}
%
\documentclass[
]{article}
\usepackage{amsmath,amssymb}
\usepackage{lmodern}
\usepackage{ifxetex,ifluatex}
\ifnum 0\ifxetex 1\fi\ifluatex 1\fi=0 % if pdftex
  \usepackage[T1]{fontenc}
  \usepackage[utf8]{inputenc}
  \usepackage{textcomp} % provide euro and other symbols
\else % if luatex or xetex
  \usepackage{unicode-math}
  \defaultfontfeatures{Scale=MatchLowercase}
  \defaultfontfeatures[\rmfamily]{Ligatures=TeX,Scale=1}
\fi
% Use upquote if available, for straight quotes in verbatim environments
\IfFileExists{upquote.sty}{\usepackage{upquote}}{}
\IfFileExists{microtype.sty}{% use microtype if available
  \usepackage[]{microtype}
  \UseMicrotypeSet[protrusion]{basicmath} % disable protrusion for tt fonts
}{}
\makeatletter
\@ifundefined{KOMAClassName}{% if non-KOMA class
  \IfFileExists{parskip.sty}{%
    \usepackage{parskip}
  }{% else
    \setlength{\parindent}{0pt}
    \setlength{\parskip}{6pt plus 2pt minus 1pt}}
}{% if KOMA class
  \KOMAoptions{parskip=half}}
\makeatother
\usepackage{xcolor}
\IfFileExists{xurl.sty}{\usepackage{xurl}}{} % add URL line breaks if available
\IfFileExists{bookmark.sty}{\usepackage{bookmark}}{\usepackage{hyperref}}
\hypersetup{
  pdftitle={Impact of the environmental drivers of condition on Biomass Prediction for sea scallop (Placopecten magellanicus) on Georges Bank, Canada},
  hidelinks,
  pdfcreator={LaTeX via pandoc}}
\urlstyle{same} % disable monospaced font for URLs
\usepackage[margin=1in]{geometry}
\usepackage{longtable,booktabs,array}
\usepackage{calc} % for calculating minipage widths
% Correct order of tables after \paragraph or \subparagraph
\usepackage{etoolbox}
\makeatletter
\patchcmd\longtable{\par}{\if@noskipsec\mbox{}\fi\par}{}{}
\makeatother
% Allow footnotes in longtable head/foot
\IfFileExists{footnotehyper.sty}{\usepackage{footnotehyper}}{\usepackage{footnote}}
\makesavenoteenv{longtable}
\usepackage{graphicx}
\makeatletter
\def\maxwidth{\ifdim\Gin@nat@width>\linewidth\linewidth\else\Gin@nat@width\fi}
\def\maxheight{\ifdim\Gin@nat@height>\textheight\textheight\else\Gin@nat@height\fi}
\makeatother
% Scale images if necessary, so that they will not overflow the page
% margins by default, and it is still possible to overwrite the defaults
% using explicit options in \includegraphics[width, height, ...]{}
\setkeys{Gin}{width=\maxwidth,height=\maxheight,keepaspectratio}
% Set default figure placement to htbp
\makeatletter
\def\fps@figure{htbp}
\makeatother
\setlength{\emergencystretch}{3em} % prevent overfull lines
\providecommand{\tightlist}{%
  \setlength{\itemsep}{0pt}\setlength{\parskip}{0pt}}
\setcounter{secnumdepth}{-\maxdimen} % remove section numbering
\usepackage{tikz} \usepackage{pdflscape}
\newcommand{\blandscape}{\begin{landscape}}
\newcommand{\elandscape}{\end{landscape}}
\newcommand{\beginsupplement}{\setcounter{table}{0}  \renewcommand{\thetable}{S\arabic{table}} \setcounter{figure}{0} \renewcommand{\thefigure}{S\arabic{figure}}}
\ifluatex
  \usepackage{selnolig}  % disable illegal ligatures
\fi

\title{Impact of the environmental drivers of condition on Biomass Prediction for sea scallop (\emph{Placopecten magellanicus}) on Georges Bank, Canada}
\author{David M. Keith\textsuperscript{1},
Jessica A. Sameoto\textsuperscript{1},
Xiaohan Liu\textsuperscript{2},
Emmanuel Devred\textsuperscript{1}, and
Catherine Johnson\textsuperscript{1}}
\date{}

\begin{document}
\maketitle
\begin{abstract}
Oceanographic conditions are known to influence natural fluctuations in fish stocks, however disentangling environmental variability from fishing effects remains challenging despite significant efforts to improve science advice through an ecosystem approach to fisheries management (EAFM). For the world's largest wild scallop fisheries, the sea scallop (\emph{Placopecten magellanicus}) found off the northeastern United States and eastern Canada annual variations in growth underlie major fluctuations in catch rate and yield. Inter-annual variability in the relative size of the harvested meat is measured using an allometric relationship between scallop meat-weight and shell height, known as scallop condition (SC), which is used to estimate the population biomass. When estimating the biomass for the current fishing year SC is an unknown parameter which is predicted based on a biological-only model. The purpose of this study was to investigate how using a recently developed model to predict SC using winter sea surface temperature (SST) impacts the prediction of current year biomass. A retrospective analysis was undertaken in which the current year biomass estimates from the stock assessment model were compared using the currently implemented bioloigical-only model and two SC-SST models. The results of this retrospective analysis indicated that there is a small positive bias of approximately 2\% in the model biomass predictions using the biological-only model. The biological-only model tends to predict that SC will be lower than the SC-SST model, thus the use of the SC-SST model results in a slight statistically insignificant increase in the bias of the model biomass predictions. The underestiamte of SC in the biological model helps to offset the tendency for the assessment model to overestimate biomass in the current fishing year. These results highlight the challenges of attempting to incorporate environmental information into existing stock assessment frameworks and suggests that the development of next generation integrated stock assessment frameworks are needed to quantiatively develop science advice that can operationlize EAFM into the science advice.
\end{abstract}

\newpage

\hypertarget{ref-sup}{%
\section{SUPPLEMENT 1}\label{ref-sup}}

\setcounter{table}{0}  \renewcommand{\thetable}{S\arabic{table}} \setcounter{figure}{0} \renewcommand{\thefigure}{S\arabic{figure}}

\newpage
\begin{landscape}
\begin{figure}
\includegraphics[width=0.75\linewidth]{D:/github/Paper_2_SDMs/Results/Figures/mesh} \caption{tester}\label{fig:PCA}
\end{figure}

\end{landscape}

\newpage

\end{document}
